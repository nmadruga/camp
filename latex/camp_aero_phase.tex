An {\ttfamily \mbox{\hyperlink{structcamp__aero__phase__data_1_1aero__phase__data__t}{camp\+\_\+aero\+\_\+phase\+\_\+data\+::aero\+\_\+phase\+\_\+data\+\_\+t}}} object describes a distinct chemical phase within an aerosol. It is designed to allow the implementation of the chemical and mass transfer processes to be independent of the particular \mbox{\hyperlink{camp_aero_rep}{aerosol representation}} used (e.\+g., bins, modes, single particles).

A single \mbox{\hyperlink{camp_aero_phase}{aerosol phase}} may be present in several \mbox{\hyperlink{camp_aero_rep}{aerosol representations}} (e.\+g., an aqueous phase in a binned and a single-\/particle representation), but the \mbox{\hyperlink{camp_species}{chemical species}} associated with a particular phase are constant throughout the model run. Once loaded, \mbox{\hyperlink{camp_aero_phase}{aerosol phases}} are made available to any \mbox{\hyperlink{input_format_aero_rep}{aerosol representations}} that want to implement them. \mbox{\hyperlink{camp_aero_rep}{Aerosol representations}} are able to specify which phases they implement and how many instances of that phase are present in the \mbox{\hyperlink{camp_aero_rep}{aerosol representation}}. For example, a binned representation with 10 bins may implement 10 aqueous phases and 10 organic phases, whereas a single particle representation with a concentric shell structure of 3 layers may implement 3 of each phase (assuming the chemistry is solved for each particle individually).

The set of \mbox{\hyperlink{camp_aero_phase}{aerosol phases}} is made available to the \mbox{\hyperlink{camp_mechanism}{mechanism(s)}} during model intialization. Reactions in the chemical mechanism are able to specify, by name, the phase in which they take place, and which species in that phase are involved. (How they decide this is up to the particular \mbox{\hyperlink{camp_rxn}{reaction type}}.) Any physical aerosol parameters, such as the surface area between phases, the particle radius, or the number concentration, required by a chemical reaction will be provided by the \mbox{\hyperlink{camp_aero_rep}{aerosol representation}} at run time.

The input format for an aerosol phase can be found \mbox{\hyperlink{input_format_aero_phase}{here}}. 